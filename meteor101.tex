\documentclass{beamer}

\usepackage{easyturk}
\usepackage[utf8]{inputenc}
\usepackage{hyperref}
\usepackage{colortbl}
\usepackage{amsmath}

\setbeamertemplate{items}[ball]
\setbeamertemplate{blocks}[rounded][shadow=true]
\setbeamertemplate{navigation symbols}{}
\usetheme{default}

\title[Meteor 101]{Meteor \.{I}le Uygulama Geli\c{s}tirme}
\author{Zekeriya Ko\c{c}}
\institute{Metglobal}
\date{Kasim 30, 2013}

\begin{document}

    \begin{frame}
        \titlepage{}
    \end{frame}

    \begin{frame}{Meteor}
        \begin{itemize}
            \item Web Geliştirme Platformu.
            \item NodeJS
            \item Yenilikçi Yaklaşım.
            \item İyi yatırım, harika bir alan adı.
            \item Tam zamanlı çalışan, kaliteli bir ekip.
        \end{itemize}
    \end{frame}

    \begin{frame}{Meteor: 7 Prensip}
        \begin{enumerate}
            \item Data iletimi. Sunucu-istemci arasında HTML değil veri gidiyor. Veriyi Render işlemi istemcinin işi.
            \item Tek dil. Hem sunucu hem istemci kodu Javascript.
            \item Veritabanı heryerde. Sunucu ve istemci taraflarında aynı arayüz.
            \item Latency Compensation. İstemciye veri önyüklemesi ve istemcide model simulasyonu.
            \item Öntanımlı tepkisellik. Öntanımlı davranış şekli tepkisel.
            \item Topluluğu kucakla. Mevcut açık kaynak projeler ile entegre.
            \item Basitlik === Verimlilik.
        \end{enumerate}
    \end{frame}

    \begin{frame}{Meteor: Proje Yap{\i}s{\i}}
        \begin{itemize}
            \item Oldukça esnek, serbest.
            \item Sunucu
                \begin{itemize}
                    \item Node.JS
                    \item Fiber --- Istek başına bir thread.
                    \item client, public, private klasörleri hariç hepsini NodeJS yükler.
                    \item private --- Assets API
                \end{itemize}
            \item İstemci
                \begin{itemize}
                    \item Browser
                    \item server, public, private klasörleri hariç tüm JS dosyalar.
                    \item Birleştirilip gönderiliyor.
                    \item Ister tek bir JS dosyası, ister klasör klasör altına. Serbest.
                \end{itemize}
            \item Ortak
                \begin{itemize}
                    \item client, server, test klasörleri hariç her şey hem sunucuda hem istemcide.
                \end{itemize}
        \end{itemize}
    \end{frame}

    \begin{frame}{Meteor: Proje Yapısı (2)}
        \begin{itemize}
            \item HTML
                \begin{itemize}
                    \item head, body, template
                    \item head, body birleştilir.
                    \item template etiketleri ise JS fonksiyonlarına çevrilir.
                \end{itemize}
            \item Yükleme sırası önemli
                \begin{itemize}
                    \item önce /lib klasörü
                    \item en son main.*
                    \item çeşitli kurallar, kaideler vs.
                \end{itemize}
        \end{itemize}
    \end{frame}

    \begin{frame}{Meteor: Veri ve G\"uvenlik}
        \begin{itemize}
            \item MongoDB --- MiniMongo
            \item Collections
            \item Publish --- Subscribe
            \item Latency compensation
            \item Methods
        \end{itemize}
    \end{frame}

    \begin{frame}{Meteor: Tepkisellik (?) --- Reactivity}
        \begin{itemize}
            \item Bağımlılık takibi.
            \item Tepkisel veri kaynakları;
                \begin{itemize}
                    \item Session
                    \item Veritabanı sorguları
                    \item vs.
                \end{itemize}
        \end{itemize}
    \end{frame}

    \begin{frame}{Meteor: Canl{\i} HTML}
    \end{frame}

    \begin{frame}{Meteor: \c{S}ablonlar --- Templates}
    \end{frame}

    \begin{frame}{Meteor: Paketler}
    \end{frame}

    \begin{frame}{Meteor: Yay{\i}nlama --- Deployment}
    \end{frame}

    \begin{frame}{Meteor: Kaynaklar}
        \begin{itemize}
            \item \url{http://www.meteor.com/}
            \item \url{http://www.discovermeteor.com}
        \end{itemize}
    \end{frame}

\end{document}
