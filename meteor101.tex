\documentclass{beamer}

\usepackage{easyturk}
\usepackage[utf8]{inputenc}
\usepackage{hyperref}
\usepackage{colortbl}
\usepackage{amsmath}

\setbeamertemplate{items}[ball]
\setbeamertemplate{blocks}[rounded][shadow=true]
\setbeamertemplate{navigation symbols}{}
\usetheme{CambridgeUS}

\title[Meteor 101]{Meteor \.{I}le Uygulama Geli\c{s}tirme}
\author{Zekeriya Ko\c{c}}
\institute{Metglobal}
\date{Kasim 30, 2013}

\begin{document}

    \begin{frame}
        \titlepage{}
    \end{frame}

    \begin{frame}{Meteor}
        \begin{itemize}
            \item Web Geliştirme Platformu.
            \item NodeJS
            \item Yenilikçi Yaklaşım.
            \item İyi yatırım, harika bir alan adı.
            \item Tam zamanlı çalışan, kaliteli bir ekip.
        \end{itemize}
    \end{frame}

    \begin{frame}{Meteor: 7 Prensip}
        \begin{enumerate}
            \item Data iletimi. Render işlemi istemcinin işi.
            \item Tek dil. Hem sunucu hem istemci kodu tamamen Javascript.
        \end{enumerate}
    \end{frame}

    \begin{frame}{Meteor: Proje Yap{\i}s{\i}}
    \end{frame}

    \begin{frame}{Meteor: Veri ve G\"uvenlik}
    \end{frame}

    \begin{frame}{Meteor: Tepkisellik (?) --- Reactivity}
    \end{frame}

    \begin{frame}{Meteor: Canl{\i} HTML}
    \end{frame}

    \begin{frame}{Meteor: \c{S}ablonlar --- Templates}
    \end{frame}

    \begin{frame}{Meteor: Paketler}
    \end{frame}

    \begin{frame}{Meteor: Yay{\i}nlama --- Deployment}
    \end{frame}

    \begin{frame}{Meteor: Kaynaklar}
        \begin{itemize}
            \item \url{http://www.meteor.com/}
            \item \url{http://www.discovermeteor.com}
        \end{itemize}
    \end{frame}

\end{document}
